\documentclass[brazil,times]{abnt}
\usepackage[T1]{fontenc}
\usepackage[utf8]{inputenc}
\usepackage{url}
\usepackage{graphicx}
\usepackage[pdfborder={0 0 0}]{hyperref}
\usepackage{amssymb}
\makeatletter
\usepackage{babel}
\makeatother

\begin{document}

\autor{Pedro Paulo Vezzá Campos \\ Ana Luísa Losnak \\ Daniel Moraes Huguenin}

\titulo{Implementação de Provador de Teoremas Usando Tableaux Analíticos}

\comentario{Primeiro exercício-programa apresentado para avaliação na disciplina
MAC0239, do curso de Bacharelado em Ciência da Computação, turma 45, da
Universidade de São Paulo, ministrada pela professora Ana Cristina Vieira de
Melo.}

\instituicao{Departamento de Ciência da Computação \par Instituto de Matemática
e Estatística \par Universidade de São Paulo}

\local{São Paulo - SP, Brasil}

\data{\today}

\capa

\folhaderosto

% \tableofcontents \chapter{}
\section*{Introdução} Para o primeiro exercício-programa (EP) de MAC0239 foi
proposto pela professora a implementação de um provador de teoremas descritos
em lógica proposicional utilizando o método dos tableaux analíticos em linguagem
Java ou C. Este relatório apresenta as estruturas e estratégias utilizadas na
implementação, além das dificuldades encontradas na concepção do programa.
	
Este relatório está organizado da seguinte forma: Primeiramente será apresentada
uma breve descrição do método dos tableaux analíticos. Depois será
apresentada a organização do programa implementado. Em seguida serão
comentadas as estruturas de dados utilizadas. Posteriormente serão detalhadas
as estratégias de implementação adotadas pelos alunos. Ainda, serão
apresentadas as formas de teste aplicadas ao programa. Por fim serão
vistas as principais dificuldades enfrentadas pelo alunos durante o projeto e
codificação do EP juntamente com uma conclusão do trabalho.

\section*{Método dos Tableaux Analíticos}
	O método dos tableaux analíticos (Ou semânticos) é um sistema de prova baseado
	em refutação que possui como vantagem o fato de ser um decididor para a
	validade de um teorema, ou seja, o algoritmo sempre encerrará sua execução em
	um tempo não necessariamente exponencial retornando como resultado a validade
	de um dado teorema.
	
	Seu funcionamento é simples e essencialmente algorítmico, permitindo uma fácil
	implementação computacional.

\section*{Organização do Programa}
	O EP foi implementado utilizando o paradigma de Orientação a Objetos, o que
	permitiu uma divisão clara de tarefas entre os diferentes objetos que cooperam
	para o funcionamento do programa. As principais classes implementadas
	encontram-se descritas abaixo:
	
	\begin{description}
		\item[main.Main]  Classe que contém o método \texttt{main()} do programa, com
		a implementação da interface do usuário. É responsável por ler os teoremas
		(Um por linha) a serem provados ou refutados, tratá-los e enviá-los ao objeto
		do tipo \texttt{ProvadorTeoremas}, exibindo o resultado da prova, ``Sequente
		válido'' caso o teorema seja válido ou ``Sequente inválido. Contra-exemplo:
		\ldots'' caso contrário.
		\item[provadorTeoremas.ProvadorTeoremas] Classe ``Fachada'' do provador. Contém
		o laço principal do programa e é responsável por escolher ramos a expandir
		além de verificar o fechamento e saturação desses ramos. Por fim, realiza as
		$\alpha$-expansões e $\beta$-expansões.
		\item[provadorTeoremas.Formula] Classe responsável por representar uma fórmula
		da lógica proposicional além de \textit{parsear} uma string para um objeto
		desse tipo.
		\item[provadorTeoremas.FormulaMarcada] Classe responsável por encapsular uma
		\texttt{Formula} junto com seu valor booleano. É responsável por retornar se é
		uma fórmula $\alpha$ ou $\beta$, e retornar suas subfórmulas marcadas de
		acordo com a regra de expansão correspondente.
		\item[provadorTeoremas.Ramo] Classe responsável por encapsular os dados a
		serem armazenados na pilha de ramos conforme descrito no exemplo de
		implementação apresentado pela professora, ou seja, o tamanho do ramo no
		momento da expansão, a segunda fórmula $\beta$ da expansão e o vetor de
		regras $\beta$ ainda por serem expandidas quando o programa retornar a esse
		ramo.
		\item[provadorTeoremas.Resultado] Classe que encapsula o resultado booleano da
		prova do sequente além de possivelmente um contra-exemplo para um sequente
		falsificável.
		\item[testesUnitarios.Testes\{Formula, FormulaMarcada, ProvadorTeoremas\}]
		Classes de testes unitários de suas respectivas classes implementados com a
		ajuda da biblioteca JUnit. Responsáveis por garantir o bom funcionamento das
		partes do provador.
	\end{description}
	 

\section*{Estruturas de Dados Utilizadas}
	O EP foi implementado utilizando a linguagem Java, o que permitiu aos alunos se
	isolarem de problemas de implementação e manipulação de estruturas de dados
	básicas durante a programação. Assim, procedeu-se de maneira geral por utilizar
	o máximo de ferramentas prontas disponíveis na API do Java.
	
	A classe \texttt{Formula} foi implementada como uma árvore binária de
	operadores. Isso simplificou o código responsável por dividir uma fórmula ao
	realizar uma $\alpha$ ou $\beta$-expansão. Para construí-la partindo de uma
	\textit{string} completamente parentetizada e bem formada, como definido no
	enunciado, foi criado um algoritmo localizado no método \texttt{parsearFormula}
	que monta recursivamente a árvore da formula passada.
	
	Já a classe \texttt{ProvadorTeoremas} conta com as seguintes estruturas de
	dados: 	
	\begin{description}
		\item[\texttt{Stack<Ramo> pilhaRamos}] Os ramos empilhados após sucessivas 
		$\beta$-expansões.
		\item[\texttt{List<Boolean> betas}] O vetor (Implementado com a classe
		\texttt{ArrayList}) de regras $\beta$ ainda por expandir.
		\item[\texttt{List<FormulaMarcada> ramo}] O vetor (Implementado com a classe
		\texttt{ArrayList}) de objetos \texttt{FormulaMarcada} que representam o
		ramo atual da prova do sequente.  
	\end{description}

	Por fim, o objeto \texttt{Resposta} possui o campo \texttt{Set<FormulaMarcada>
	contraExemplo} (Implementado com a classe \texttt{HashSet}) para representar o
	contra-exemplo encontrado durante a prova de um sequente falsificável. O uso da
	estrutura de dados \texttt{Set} garante que não haverá duplicação de átomos no
	contra-exemplo apresentado ao usuário.
	
\section*{Estratégias de Implementação}
	Para a expansão do ramo foi utilizada a estratégia de realizar o máximo de
	$\alpha$-expansões possível antes de proceder com $\beta$-expansões. Isso
	trouxe como benefício uma menor quantidade de ramos divididos no tableau
	aumentando a velocidade da prova.
	
	Para a escolha de qual $\beta$-expansão aplicar foi escolhida a estratégia
	ascendente, que opta pela última regra $\beta$ ainda por expandir no vetor
	\texttt{betas}.
	
	De forma a economizar tempo e espaço computacional o provador remove regras
	$\alpha$ já expandidas trazendo maior facilidade na análise manual do
	funcionamento do provador e maior velocidade do programa já que a verificação
	do fechamento e saturação dos ramos devem iterar menos elementos por vez.


\section*{Testes}
	O bom funcionamento do programa foi verificado através de diversos testes
	unitários implementados com a biblioteca JUnit. Como consequência, a cada
	alteração no código fonte foi possível detectar erros de codificação de maneira
	fácil e rápida.
	
	O \textit{parseamento} de fórmulas foi testado com 24 testes diferentes
	enquanto o funcionamento do provador com um todo foi verificado com 56 testes
	distintos. Os testes unitários podem ser vistos no diretório
	\texttt{testesUnitarios} junto com o código fonte do programa.
	
	Dentre os testes produzidos incluem-se: \textit{Modus ponens}, \textit{Modus
	tollens}, Leis de De Morgan, contrapositiva, converso, silogismos hipotético e
	disjuntivo, resolução, adição, conjunção, simplificação, além de diversas
	tautologias clássicas tais como distributividade e comutatividade.
	
	Por fim, a interface com o usuário também foi testada. As entradas aplicadas
	podem ser vistas no diretório \texttt{testes/Entradas.txt} e as saídas esperadas em
	\texttt{testes/Saidas.txt}.

\section*{Dificuldades Encontradas}
	O \textit{parseamento} de fórmulas foi implementado inicialmente com um
	algoritmo iterativo que mostrou-se bastante frágil com casos de testes mais
	complexos. Assim, como trata-se de uma parte essencial ao bom funcionamento do
	programa optou-se por refazê-lo de maneira recursiva, o que trouxe ganhos de
	legibilidade de código além de maior robustez.
	
	Ainda, outro problema encontrado foi garantir que o provador estava retornando
	resultados corretos para os sequentes. Inicialmente o provador foi implementado
	de maneira que à medida que uma fórmula fosse consumida por uma expansão ela
	seria removida do ramo. Isso facilitou o \textit{debug} do programa
	inicialmente e funcionava corretamente para casos de teste menores mas
	provou-se incorreta para certos sequentes. Para corrigir isso, os alunos
	perceberam que preservando as fórmulas $\beta$ no ramo esse problema seria
	resolvido.

\section*{Conclusão}
	Este primeiro EP de MAC0239 proposto pela professora Ana foi de
	grande valia por apresentar como proposta a implementação de um provador
	usando tableaux analíticos visto em aula, fixando conceitos e detalhes
	computacionais relevantes para futuros profissionais da Ciência da Computação.

	Ainda, este trabalho contribuiu ao resumir bem diversos conceitos vistos
	durante o semestre na área de prova de teoremas até culminar no
	projeto, desenvolvimento e testes de uma aplicação prática. Por outro lado,
	como os exercícios não demandaram grandes e tediosos esforços de programação os
	acadêmicos puderam concentrar-se nas modelagens e abordagens adotadas para
	resolver o problema, as quais foram apresentadas anteriormente neste relatório.


% \bibliographystyle{abnt-num}
% \bibliography{bibliografia}
\end{document}